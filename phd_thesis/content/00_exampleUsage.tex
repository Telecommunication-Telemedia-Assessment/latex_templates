\chapter{Introduction}
%\chapter{Einleitung}
\label{sec:introduction}

general motivation for your work, context and goals: 1-2 pages

\begin{itemize}
    \item \textbf{Context:} make sure to link where your work fits in
    \item \textbf{Problem:} gap in knowledge, too expensive, too slow, a deficiency, superseded technology
    \item \textbf{Strategy:} the way you will address the problem
\end{itemize}


\section{Sample Section}

The following samples explain how to insert references, figures and tables, how to set math, algorithms and program code.


\begin{itemize}[align=left,leftmargin=*]
    \citeitem{raake2014quality}
\end{itemize}

\subsection{Figures}

Figure~\ref{fig:ex1} shows something.

\begin{figure}
    \centering
    \missingfigure{missing figure}
    \caption{Example Figure}
    \label{fig:ex1}
\end{figure}

\begin{figure}
    \begin{center}
        \begin{tikzpicture}
            \tikzstyle{every entity}=[fill=blue!20,draw=blue,thick]
            \tikzstyle{every relationship}=[fill=orange!20,draw=orange,thick,aspect=1.5]

            \node[entity] (sheep) at (0,0) {Sheep};
            \node[entity] (genome) at (4,0) {Genome};
            \node[relationship] at (2,1.5) {has}
                edge (sheep)
                edge (genome);
            \draw[->] (sheep) -- (genome);
        \end{tikzpicture}
    \end{center}
    \caption{Tikz Example}
    \label{fig:ex2}
\end{figure}
You can also use tikz for your figures, see Figure~\ref{fig:ex2}.
However other tools, like \url{https://www.yworks.com/products/yed}, \url{draw.io} are more suitable.

\subsection{Tables}

Short tables, e.g., Table~\ref{tab:shorttable} are straighforward to define.

\begin{table}
    \centering
    \begin{tabular}{llr}
        \toprule
        left aligned & same here & right aligned \\
        \midrule
        1 & 2 & 3 \\
        4 & 5 & 6 \\
        7 & 8 & 9 \\
        \bottomrule
    \end{tabular}
    \caption{Short table}
    \label{tab:shorttable}
\end{table}

Multi-line cells can be set as shown in Table~\ref{tab:SensorNetworkApplications}.

\begin{table}
    \centering
    \begin{tabular}{>{\raggedright}p{1.8cm}p{5.4cm}p{3.4cm}}
        \toprule
        Class & application examples & lifetime aspects \\
        \midrule
        Critical, coverage &
                Forest fire detection, flood detection, nuclear/chemical/biological attack detection, battlefield surveillance, intrusion detection &
                $c_{ca}$/$c_{ct}$/$c_{cb}$, $c_{ln}$, $c_{la}$, $c_{lo}$\\
        Critical, no coverage &
                Monitoring human physiological data, military monitoring of friendly forces, machine monitoring &
                $c_{cc}$, $c_{ln}$, $c_{la}$, $c_{lo}$ \\
        Noncritical, coverage &
                Agriculture, smart buildings, habitat monitoring (sensors monitor the inhabitants in a region) &
                $c_{ac}$/$c_{tc}$/$c_{bc}$, $c_{cc}$, $c_{sd}$ \\
        Noncritical, no coverage &
                Home automation, habitat monitoring (sensors are attached to animals and monitor their health and social contacts) &
                $c_{cc}$, $c_{sd}$ \\
        \bottomrule
    \end{tabular}
    \caption{Sensor network applications}
    \label{tab:SensorNetworkApplications}
\end{table}


\subsection{Math}

Simple inlined equations: $\zeta(t) = \min( \zeta_{**}(t))$.
The same in a numbered equation, i.e.\ Eq.~\ref{eq:zeta}, which states
\begin{equation}
\zeta(t) =
    \min\left(
        \zeta_{**}(t)
    \right)
\label{eq:zeta}
.
\end{equation}

\begin{equation}
    e = m \cdot c^2
    \label{eq:ex1}
\end{equation}


\[ v = \frac{s}{t} \]

\begin{equation}
    \ln(e) + \sin^2(p) + \cos^2 (p) = \sum_{n=0}^{\infty} \left(\frac{1}{2}\right)^n
    \label{eq:ex2}
\end{equation}


Equations covering multiple lines should be aligned. Note that the numbering is added automatically, independent of whether the equation is actually referenced or not, as in
\begin{align}
sd_{max} &=
    \max\left(
        (t_{i+1} - t_i)
            : \zeta(t_i) < 1, i \in [0, |T|-1]
    \right)
,\\
\psi_{sd}(t) &=
    \begin{cases}
        \dfrac{\Delta t_{sd}}{sd_{max}}
            & \text{if $sd_{max} > 0$}, \\
        1
            & \text{if $sd_{max} = 0$},
    \end{cases}
\\
\zeta_{sd}(t) &=
    \frac{
        \psi_{sd} - cl_{sd}
    }{
        c_{sd} - cl_{sd}
    }
.
\end{align}


\subsection{Algorithms}
Algorithm~\ref{algo:test} is an example algorithm.

\begin{algorithm}[H]
    \KwData{this text}
    \KwResult{how to write algorithm with \LaTeX2e }
    initialization\;
    \While{not at end of this document}{
        read current\;
        \eIf{understand}{
            go to next section\;
            current section becomes this one\;
        }{
            go back to the beginning of current section\;
        }
    }
    \caption{How to write algorithms}
    \label{algo:test}
\end{algorithm}

\subsection{Program Code}

Program code should be omitted, but if absolutely necessary, it should be set as seen in Listing~\ref{lst:code}.

\begin{lstlisting}[caption=Sample application,label=lst:code]{}
#include <iostream>
int main() {
    std::cout << "Hello world" << std::endl;
    return 0;
}

\end{lstlisting}


\subsection{References}

You can cite something with~\cite{Berlin}.
Or~\citeauthor{raake2014quality} for Authors~\cite{raake2014quality}.
Or~\cite{coresparql}. Or \cite[p 1]{sparqlAlgebra}

For bibtex entries you can use different sources, e.g.,
\begin{itemize}
    \item \url{http://scholar.google.de/}
    \item \url{http://books.google.de}
    \item \url{http://citeseerx.ist.psu.edu}
    \item \url{http://ieeexplore.ieee.org/}
    \item \url{http://dblp.uni-trier.de/}
\end{itemize}

\subsection{TODOs and FIXMEs}

You can use the the \verb|\todoI| command to add short ``sticky notes'' to your document.
\todoI{This is what a TODO looks like}
This will also trigger generation of a list-of-TODOs at the end of the document.
The same goes for the \verb|\note|\note{This is what a NOTE looks like} command.




\chapter{Fundamentals}
\label{sec:fundamentals}

Fundamentals / environment and related work: 1/3

\begin{itemize}
    \item comment on employed hardware and software
    \item describe methods and techniques that build the basis of your work
    \item review related work(!)
\end{itemize}

\chapter{Architecture/Implementation}

Developed architecture / system design / implementation: 1/3

\begin{itemize}
    \item start with a theoretical approach
    \item describe the developed system/algorithm/method from a high-level point of view
    \item go ahead in presenting your developments in more detail
\end{itemize}


\chapter{Analysis/ Evaluation}

Measurement results / analysis / discussion: 1/3

\begin{itemize}
    \item whatever you have done, you must comment it, compare it to other systems, evaluate it, using e.g. subjective tests
    \item usually, adequate graphs help to show the benefits of your approach
    \item caution: each result/graph must be discussed! what's the reason for this peak or why have you ovserved this effect
\end{itemize}


\chapter{Conclusion}

Conclusion: 1 page

\begin{itemize}
    \item summarize again what your thesis did, but now emphasize more the results, and comparisons
    \item write conclusions that can be drawn from the results found and the discussion presented in the paper
    \item future work (be very brief, explain what, but not much how)
\end{itemize}


\begin{leftbar}
Lorem ipsum dolor sit amet, consectetur adipisicing elit, sed do eiusmod
tempor incididunt ut labore et dolore magna aliqua. Ut enim ad minim veniam,
quis nostrud exercitation ullamco laboris nisi ut aliquip ex ea commodo
consequat. Duis aute irure dolor in reprehenderit in voluptate velit esse
cillum dolore eu fugiat nulla pariatur. Excepteur sint occaecat cupidatat non
proident, sunt in culpa qui officia deserunt mollit anim id est laborum.
\end{leftbar}